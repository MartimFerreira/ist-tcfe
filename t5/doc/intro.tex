\section{Introduction}
\label {sec:introduction}

% objective
% circuit description
% explain why we decided to call the ground node 0 instead of 4
% circuit image
% --- present the values that were given to us
% --- talk about the significant digits
% sections

% state the learning objective 
In this laboratory assignment it was our aim to implement a bandpass filter with specific characteristics: a central frequency of 1kHz and a gain at central frequency of 40dB. In order to do that, we had available one 741 OPAMP and a maximum of three of each of the following components: 1k$\Omega$, 10k$\Omega$ and 100k$\Omega$ resistors, and 220nF and 1$\mu$F capacitors. Our objective was also to have the lowest cost and gain and central frequency deviations.

The circuit used was the one in Figure \ref{fig:circuit}. Its characteristics are in \ref{tab:circuit}.

% descrever mais o circuito?

% PÔR O CIRCUITO
\begin{figure}[H] \centering
\includegraphics[width=1.0\linewidth]{rc.pdf}
\caption{The circuit in analysis}
\label{fig:circuit}
\end{figure}


\begin{table}[H]
  \centering
  \begin{tabular}{|c|c|}
    \hline
      \input{components_tab.tex}
  \end{tabular}
  \caption{Circuit characteristics (resistances and capacitances)}
  \label{tab:circuit}
\end{table}


%\begin{table}[H]
%  \centering
%  \begin{tabular}{|c|c|}
%    \hline
%      \input{resistance1_tab.tex}
%  \end{tabular}
%  \caption{Transistors' characteristics}
%  \label{tab:resistance1}
%\end{table}

The ~\ref{sec:analysis} and ~\ref{sec:simulation} sections have the theoretical analysis and the ngspice simulation of the circuit. In ~\ref{sec:conclusion} there is the conclusion of the report.


~\ref{sec:conclusion}
