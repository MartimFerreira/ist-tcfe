\section{Theoretical Analysis}
\label{sec:analysis}


% MUITO IMPORTANTE: LER OS COMENTÁRIOS

\subsection{Input and ouput impedances and gain}

One should first note that the capacitors' impedances are given by $Z_{C_1} = \frac{1}{j \omega C_1}$ and $Z_{C_2} = \frac{1}{j \omega C_2}$.


% ------------------------ NÃO SEI SE AS IMPEDÂNCIAS E A EXPLICAÇÃO DAS EXPRESSÕES ESTÃO BEM ---------------------------------

From the input, capacitor $C_1$ and resistor $R_1$ are seen as being in series. Meanwhile, from the output, resistor $R_2$ and capacitor $C_2$ are seen as being in parallel for intermediate frequencies, which is the range used in this assignment. Therefore, one has

\begin{equation}
Z_{in} = |Z_{C_1} + (R_1||R_6)|
\end{equation}

\begin{equation}
Z_{out} = \left|\frac{1}{\frac{1}{R_2 || R_5} + \frac{1}{Z_{C_2}}}\right|
\end{equation}

These are the impedances at the central frequency. Note that $R_1||R_6 = \frac{R_1 R_6}{R_1 + R_6}$ and $R_2||R_5 = \frac{R_2 R_5}{R_2 + R_5}$.



%As for the gain, from the input of the amplification part of the circuit, we have $V_p = \frac{R_1}{R_1 + Z_{C_1}}$, whereas, for the non-inverting amplifier alone, the gain would be equal to $1 + \frac{R_3}{R_4}$, multiplying these two, $amp = \left(1 + \frac{R_3}{R_4}\right) V_p$.

As for the gain, one should remember that the circuit is made up of three parts: a high pass filter, a non-inverting amplifier and a low pass filter, each with a certain gain given by the ratio $\frac{V_{out}}{V_{in}}$, where $V_{out}$ and $V_{in}$ are the output and input voltages for each stage. Their gains are, respectively: $Gain_{HPF} = \frac{R_1}{R_1 + Z_{C_1}}$, $Gain_{amp} = 1+\frac{R_3}{R_4}$ and $Gain_{LPF} = \frac{Z_{C_2}}{Z_{C_2}+(R_2||R_5)}$, as studied in theoretical lectures. Multiplying these three values, one obtains the total gain for the circuit:

\begin{equation}
Gain = \left| \frac{R_1||R_6}{(R_1||R_6) + Z_{C_1}} \times  \left(1+\frac{R_3}{R_4}\right) \times \frac{Z_{C_2}}{Z_{C_2}+(R_2||R_5)} \right|
\end{equation}


% ------------------------ NÃO SEI SE AS FREQUÊNCIAS E A EXPLICAÇÃO DAS EXPRESSÕES ESTÃO BEM ---------------------------------

In respect to the low and high cutoff frequencies, they are given by 

\begin{equation}
f_{low \ cutoff} = \frac{1}{2 \pi (R_1||R_6) C_1}
\end{equation}

\begin{equation}
f_{high \ cutoff} = \frac{1}{2 \pi (R_2||R_5) C_2}
\end{equation}

Finally, the central frequency can be calculated using the following expression:

\begin{equation}
f_{central} = \sqrt{f_{low \ cutoff} \times f_{high \ cutoff}}
\end{equation}

In the table below we have the results obtained in Octave for this part:

\begin{table}[H]
  \centering
  \begin{tabular}{|c|c|}
    \hline
      \input{results_tab.tex}
  \end{tabular}
  \caption{Results obtained in the theoretical analysis. The gain is given in linear units.}
  \label{tab:results}
\end{table}

%It is important to mention that non-ideal characteristics from the OPAMP based on the real model were used. Ideally the OPAMP would have $Z_{in} = \infty$ and $Z_{out} = 0$. However, if we assume that these parameters as well as the gain are finite, it is possible to determine them by taking measures in the frequency. The gain should be real and quite high and $Z_{in}$ and $Z_{out}$ should both be complex impedances.




\subsection{Frequency response}

In Figure \ref{fig:freq_response} we can see a graph of the total circuit gain (in dB) as a function of frequency.


\begin{figure}[H] \centering
\includegraphics[width=0.7\linewidth]{freq.eps}
\caption{Frequency response, $\frac{V_o(f)}{V_i(f)}$}
\label{fig:freq_response}
\end{figure}

\begin{figure}[H] \centering
\includegraphics[width=0.7\linewidth]{angle.eps}
\caption{Frequency response - Phase}
\label{fig:freq_response_phase}
\end{figure}

Here we can confirm that the circuit is indeed a bandpass filter with an approximately 40dB gain in the bandpass and for which $f = 1 kHz$ is part of the passband. Inspecting the phase frequency response graph we see that the phase shift goes from 90 \degree, below the center frequency to 0 \degree at the center frequency and then to  –90\degree above the center frequency. The shape of this curve is characteristic of a bandpass filter. 





%\begin{figure}[H] \centering
%\includegraphics[width=0.6\linewidth]{v_out_plot.eps}
%\caption{Output voltage for Envelope Detector (v(mid1), in red) and Voltage Regulator (v(out1), in blue)}
%\label{fig:octave_centered_output}
%\end{figure}
