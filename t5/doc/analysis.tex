\section{Theoretical Analysis}
\label{sec:analysis}


% MUITO IMPORTANTE: LER OS COMENTÁRIOS

\subsection{Input and ouput impedances and gain}

One should first note that the capacitors' impedances are given by $Z_{C_1} = \frac{1}{j \omega C_1}$ and $Z_{C_2} = \frac{1}{j \omega C_2}$.

From the input, capacitor $C_1$ and resistor $R_1$ are seen as being in series. Meanwhile, from the output, resistor $R_2$ and capacitor $C_2$ are seen as being in parallel for intermediate frequencies, which is the range used in this assignment. Therefore, one has

\begin{equation}
Z_{in} = |Z_{C_1} + R_1|
\end{equation}

\begin{equation}
Z_{out} = |\frac{1}{\frac{1}{R_2} + \frac{1}{Z_{C_2}}}|
\end{equation}

%As for the gain, from the input of the amplification part of the circuit, we have $V_p = \frac{R_1}{R_1 + Z_{C_1}}$, whereas, for the non-inverting amplifier alone, the gain would be equal to $1 + \frac{R_3}{R_4}$, multiplying these two, $amp = \left(1 + \frac{R_3}{R_4}\right) V_p$.
As for the gain, one should remember that the circuit is made up of three parts: a high pass filter, a non-inverting amplifier and a low pass filter, each which a certain gain, given by the ration $\frac{V_{out}}{V_in}$, where $V_{out}$ and $V_{in}$ are the output and input voltages for each stage. That being said, their gains are, respectivelly: $Gain_{HPF} = \frac{R_1}{R_1 + Z_{C_1}}$, $Gain_{amp} = 1+\frac{R_3}{R_4}}$ and $Gain_{LPF} = \frac{Z_{C_2}}{Z_{C_2}+R_2}$, as studied in theoretical lectures. Multiplying these three values, one obtains the total gain for the circuit:

\begin{equation}
Gain = \frac{R_1}{R_1 + Z_{C_1}} \times  \left(1+\frac{R_3}{R_4}\right) \times \frac{Z_{C_2}}{Z_{C_2}+R_2}
\end{equation}


\subsection{Frequency response}



\begin{table}[H]
  \centering
  \begin{tabular}{|c|c|}
    \hline
      \input{resultsDC1_tab.tex}
  \end{tabular}
  \caption{Values obtained in OP analysis}
  \label{tab:resultsDC1}
\end{table}




%\begin{figure}[H] \centering
%\includegraphics[width=0.6\linewidth]{v_out_plot.eps}
%\caption{Output voltage for Envelope Detector (v(mid1), in red) and Voltage Regulator (v(out1), in blue)}
%\label{fig:octave_centered_output}
%\end{figure}
