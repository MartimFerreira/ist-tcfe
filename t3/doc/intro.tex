\section{Introduction}
\label {sec:introduction}

% objective
% circuit description
% explain why we decided to call the ground node 0 instead of 4
% circuit image
% --- present the values that were given to us
% --- talk about the significant digits
% sections

% state the learning objective 
The aim of this laboratory assignment is to build an AC-DC converter circuit whose input is AC voltage with frequency $50 \ Hz$ and amplitude $230 \ V$ and whose output is DC voltage $12 \; V$.


% corrigir, porque provavelmente há coisas que não estão certas, por causa das alterações que fizemos depois de falar com o stor na última aula de laboratório (especialmente na parte da esquerda, antes do transformador)
% modelos usados para os díodos
% tentativa erro
The circuit is made up of several parts. In the first one there is a voltage source $v_S$, then there is a transformer with $n$ loops on the left and 1 loop on the right, followed by a full-wave bridge rectifier, an envelope detector and a voltage regulator.

The transformer was replaced by its ideal model, namely by a dependent voltage source between nodes $In1$ and $In2$ (that we named $V_2$) of value $1/n * v_S$,  and one dependent current source between nodes Begin1 and the left-side ground of value $1/n * I_2$, where $I_2$ is the current going from $In2$ to $In1$. In order to properly simulate this in ngspice, an imaginary voltage source of $O\;V$ was added in between $In2$ and the dependent voltage source.

The rectifier is composed of 4 diodes ($Dfw1$, $Dfw2$, $Dfw3$ and $Dfw4$). The two other parts were based on the examples explained in the theoretical lectures. The envelope detector has a rectifier with a diode $Ddet$ and a resistor $Rdet$ plus a capacitor $C$ in parallel with the resistor. Finally, there is the voltage regulator with a resistor $Rreg$ and a limiter made up of 4 series in parallel, each with 18 diodes (for a total of 72 diodes). The diodes in parallel help reduce the total resistance in this part of the circuit.

% componentes - done
% falta falar dos nós, maybe?
% esquema do circuito

The main objectives were to obtain an average output voltage of $12 \ V$ with a very small ripple while having the cheapest possible circuit.

\begin{figure}[H] \centering
\includegraphics[width=0.9\linewidth]{rc.pdf}
\caption{The circuit in analysis}
\label{fig:circuit}
\end{figure}

\begin{table}[H]
  \centering
  \begin{tabular}{|c|c|}
    \hline
      \input{resistance_tab.tex}
  \end{tabular}
  \caption{Resistances, n and voltage and resistance associated with the diode. .}
  \label{tab:resistance}
\end{table}


In Section~\ref{sec:analysis}, there is the theoretical analysis of the circuit (all the necessary calculations were made with \textit{Octave}). In Section~\ref{sec:simulation}, the circuit is simulated with \textit{ngspice}. The theoretical and simulation results are then compared. Lastly,  Section~\ref{sec:conclusion} is the conclusion of the report. 
