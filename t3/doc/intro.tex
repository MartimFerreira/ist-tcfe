\section{Introduction}
\label{sec:introduction}

% objective
% circuit description
% explain why we decided to call the ground node 0 instead of 4
% circuit image
% --- present the values that were given to us
% --- talk about the significant digits
% sections

% state the learning objective 
The aim of this laboratory assignment is to study the RC circuit in Figure~\ref{fig:rc}. It contains two voltage sources (a linear current controlled one, $V_d$, and an independent one, $v_s$), one linear voltage controlled  current source ($I_b$), resistors ($R_1$, $R_2$, $R_3$, $R_4$, $R_5$, $R_6$ and $R_7$) and a capacitor ($C$).
The $v_s$ voltage supply varies over time.

\[ 
v_s (t)= \left\{
\begin{array}{ll}
  V_s & t \leq 0 \\
  \sin(2 \pi f t) & t > 0\\
\end{array} 
\right. 
\]

\begin{table}[hbt!]
  \centering
  \begin{tabular}{|c|c|}
    \hline    
    {\bf Name} & {\bf Value} \\ \hline
    \input{octave_data_tab.tex}
  \end{tabular}
  \caption{Data received}
  \label{tab:data}
\end{table}



With regard to the identification of the nodes it was decided that the ground node would be referred to as node 0. Node 4 was also an option but because in \textit{Ngspice} 0 is the ground node, this notation was also used in the theoretical analysis for reasons of convenience and coherence.



\begin{figure}[h] \centering
\includegraphics[width=0.6\linewidth]{rc.pdf}
\caption{The circuit in analysis}
\label{fig:rc}
\end{figure}




%\begin{table}[hbt!]
%  \centering
%  \begin{tabular}{|c|c|c|}
%    \hline
%   & \textbf{Original Values}        & \textbf{Approximated Values ---- confirmar!!!}\\ \hline
%    \input{-------FALTA----------------}
%  \end{tabular}
%  \caption{Significant figures considered by Octave and Ngsipe- approximation} 
%  \label{tab:ap}
%\end{table}



% falar outra vez dos algarismos significativos
%The original data that was received had 11 significant digits. However, Octave, by default, works with an approximation of 5 significant digits. With this new approximation the results still present high precision and the uncertainty range is still below 1\%, so we kept it, while still using the original values in ngspice. The table \ref{tab:ap} has all the original values and the approximated values that were considered. 
% appprooximated values da tabela ainda têm que ser confirmados

% secções
It was our goal to determine how the circuit behaved for different time intervals, as well as to understand how the voltages changed (in particular, the voltage in node 6) not only over time but also with the frequency of the voltage source.

In Section~\ref{sec:analysis}, a theoretical analysis of the circuit is presented. Here all systems of equations were solved using \textit{Octave}. In Section~\ref{sec:simulation}, the circuit is simulated using \textit{ngspice}, with the results being compared to those obtained previously. Section~\ref{sec:conclusion} is the conclusion of the report. %Lastly, in Section~\ref{sec:conclusion}, the simulation and the theoretical predictions are compared and conclusions are taken.


%following a similar procedure to that of the previous section
