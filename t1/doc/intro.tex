\section{Introduction}
\label{sec:introduction}

% state the learning objective 
\par
The aim of this laboratory assignment is to study the circuit shown below assuming the lumped-elements approximation. It is composed of voltage sources (an independent one, $V_a$, and a linear current-controlled one, $V_c$), current sources (an independent one, $I_d$, and a linear voltage-controlled one, $I_b$) and resistors ($R_1$, $R_2$, $R_3$, $R_4$, $R_5$, $R_6$ and $R_7$). All components of this circuit are linear and time-invariant.

\begin{figure}[h] \centering
\includegraphics[width=0.6\linewidth]{rc.pdf}
\caption{The circuit in analysis}
\label{fig:rc}
\end{figure}

\par
It was known beforehand that:
$R_1 = 1039.3 \ \Omega$,
$R_2 = 2010.9 \ \Omega$,
$R_3 = 3041.2 \ \Omega$,
$R_4 = 4183.8 \ \Omega$,
$R_5 = 3095.8 \ \Omega$,
$R_6 = 2021.2 \ \Omega$,
$R_7 = 1022.3 \ \Omega$,
$V_a = 5.1798 \ V$,
$I_d = 0.0010044 \ A$,
$K_b = 0.0070875 \ S$ and
$K_c = 8185.8 \ \Omega$.


\par
$N_0$ was chosen as the ground node, $V_0 = 0 \ V$. $V_i$ is the voltage in node $N_i$ ($i=1, 2, ..., 7$) and $I_i$  is the current that goes through mesh $i$ counterclockwise ($i=A, B, C, D$).

\par
In Section~\ref{sec:analysis}, a theoretical analysis of the circuit is presented following two of the most common methods: the mesh method and the nodal method (whereas in the latter it is to find the potential in each node). In the mesh method, the goal is to find the values of the currents in each essential mesh. The equations written in order to do this are applications of Kirchhoff's voltage law, or KVL (which states that "the directed sum of the potential differences (voltages) around any closed loop is zero'') to essential meshes or determined by inspection of the circuit. If more equations are needed (because KVL couldn't be applied to one of the meshes), a super-mesh equation is written. In the nodal method, the goal is to determine all node potentials, with the set of equations written being based on Kirchhoff's current law, or KCL, (''the algebraic sum of currents in a network of conductors meeting at a point is zero'') applied to the nodes. If more equations are needed, super-node equations are used. In this part, Octave was used to solve all systems of (linear) equations. In Section~\ref{sec:simulation}, the circuit is analysed by simulation using Ngspice. Finally, the comparison between both types of analysis and the conclusions of this study are in Section~\ref{sec:conclusion}.

