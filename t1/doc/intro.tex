\section{Introduction}
\label{sec:introduction}

% state the learning objective 
\par
The aim of this laboratory assignment is to study the circuit shown below assuming the lumped-elements approximation. It is composed of voltage sources (an independent one, $V_a$, and a linear current-controlled one, $V_c$), current sources (an independent one, $I_d$, and a linear voltage-controlled one, $I_b$) and resistors ($R_1$, $R_2$, $R_3$, $R_4$, $R_5$, $R_6$ and $R_7$). All components of this circuit are linear and time-invariant.

\begin{figure}[h] \centering
\includegraphics[width=0.6\linewidth]{rc.pdf}
\caption{The circuit in analysis}
\label{fig:rc}
\end{figure}

\par
It was known beforehand that:
$R_1 = 1039.3 \ \Omega$,
$R_2 = 2010.9 \ \Omega$,
$R_3 = 3041.2 \ \Omega$,
$R_4 = 4183.8 \ \Omega$,
$R_5 = 3095.8 \ \Omega$,
$R_6 = 2021.2 \ \Omega$,
$R_7 = 1022.3 \ \Omega$,
$V_a = 5.1798 \ V$,
$I_d = 0.0010044 \ A$,
$K_b = 0.0070875 \ S$ and
$K_c = 8185.8 \ \Omega$.


\par
$N_0$ was chosen as the ground node, $V_0 = 0 \ V$. $V_i$ is the voltage in node $N_i$ and $I_i$ ($i=1, 2, ..., 7$) is the current that goes through mesh $i$ counterclockwise ($i=A, B, C, D$).

\par
In Section~\ref{sec:analysis}, a theoretical analysis of the circuit is presented following two of the most common methods: the mesh method and the nodal method (in the former the objective was to find the values of the current in each essential mesh, whereas in the latter it was to find the potential in each node). In Section~\ref{sec:simulation}, the circuit is analysed by simulation. Finally, the  comparison between both types of analysis and the conclusions of this study are in Section~\ref{sec:conclusion}.

