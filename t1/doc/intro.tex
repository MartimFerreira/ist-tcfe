\section{Introduction}
\label{sec:introduction}

% state the learning objective 
\par
The aim of this laboratory assignment is to study the circuit shown below assuming the lumped-elements approximation. It is composed of voltage sources (an independent one, $V_a$, and a linear current-controlled one, $V_c$), current sources (an independent one, $I_d$, and a linear voltage-controlled one, $I_b$) and resistors ($R_1$, $R_2$, $R_3$, $R_4$, $R_5$, $R_6$ and $R_7$). All components of this circuit are linear and time-invariant.

The original data that was received had 11 significant digits. However, Octave, by default, works with an approximation of 5 significant digits. With this new approximation the results still present high precision and the uncertainty range is still below 1\%, so we kept it, while still using the original values in ngspice. The table \ref{tab:ap} has all the original values and the approximated values that were considered. 

\begin{table}[hbt!]
  \centering
  \begin{tabular}{|c|c|c|}
    \hline
    &           \textbf{Original Values}        & \textbf{Approximated Values}\\ \hline
    $R_1 (\Omega)$   &  $   1039.30068064   $     & $1039.3$\\ \hline
    $R_2 (\Omega)$   &  $   2010.9087471    $      & $2010.9$\\ \hline
    $R_3 (\Omega)$   &  $   3041.24242148   $    & $3041.2$\\ \hline
    $R_4 (\Omega)$   &  $   4183.83562625   $     & $4183.8$\\ \hline
    $R_5 (\Omega)$   &  $   3095.83437009   $     & $3095.8$\\ \hline
    $R_6 (\Omega)$   &  $   2021.17084711   $     & $2021.2$\\ \hline
    $R_7 (\Omega)$   &  $   1022.26630661   $     & $1022.3$\\\hline
    $V_a (V)$   &  $   5.17979967502   $     & $5.1798$\\ \hline
    $I_d (A)$   &  $    0.00100439545365     $     & $0.0010044$\\ \hline
    $K_b (S)$   &  $    0.00708750963899     $     & $0.0070875$\\ \hline
    $K_c (\Omega)$   &     $8185.75062147    $     & $8185.8$\\ \hline
  \end{tabular}
  \caption{Significant figures considered by Octave and Ngsipe- approximation} 
  \label{tab:ap}
\end{table}

\par
$N_0$ was chosen as the ground node, $V_0 = 0 \ V$. $V_i$ is the voltage in node $N_i$ ($i=1, 2, ..., 7$) and $I_i$  is the current that goes through mesh $i$ counterclockwise ($i=A, B, C, D$).


\par
In Section~\ref{sec:analysis}, a theoretical analysis of the circuit is presented following the mesh method and the nodal method. In this part, Octave was used to solve all systems of (linear) equations. In Section~\ref{sec:simulation}, the circuit is analysed using ngspice. Finally, the comparison between both types of analysis and the conclusions of this study are in Section~\ref{sec:conclusion}.

\begin{figure}[h] \centering
\includegraphics[width=0.6\linewidth]{rc.pdf}
\caption{The circuit in analysis}
\label{fig:rc}
\end{figure}
