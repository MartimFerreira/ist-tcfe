\section{Introduction}
\label{sec:introduction}

% state the learning objective 
\par
The aim of this laboratory assignment is to study a circuit composed exclusively by voltage sources ($V_a$ and $V_c$), current sources ($I_b$ and $I_d$), as well as resistors ($R_1$, $R_2$, $R_4$, $R_5$, $R_6$ and $R_7$). That being said, this circuit, which is shown in the image below, contains only linear components.

\par
It was known beforehand that:
$R_1 = 1039.3 \ \Omega$,
$R_2 = 2010.9 \ \Omega$,
$R_3 = 30411.2 \ \Omega$,
$R_4 = 4183.8 \ \Omega$,
$R_5 = 3095.8 \ \Omega$,
$R_6 = 2021.2 \ \Omega$,
$R_7 = 1022.3 \ \Omega$,
$V_a = 5.1798 \ V$,
$I_d = 0.0010044 \ A$,
$K_b = 0.0070875 \ A/V$ and
$K_c = 8185.8 \ V/A$.

\par
Therefore, the objective was to find the values of: $V_0$, $V_1$, $V_2$, $V_3$, $V_4$, $V_5$, $V_6$, $V_7$, $I_b$, $I_c$, $I_1$, $I_2$, $I_3$, $I_4$, $V_b$ and $V_c$.

\begin{figure}[h] \centering
\includegraphics[width=0.8\linewidth]{rc.pdf}
\caption{The circuit in study}
\label{fig:rc}
\end{figure}

\par
In order to make it easier to analyse the circuit, both meshes and nodes were labelled. $N_0$ was chosen as the ground node, $V_0 = 0 \ V$. $V_i$ is the voltage in node $N_i$ and $I_1$ is the current that goes through mesh $A$ counterclockwise. Likewise, $I_2$ goes through $B$, $I_3$ through $C$ and $I_4$ through $D$, all anticlockwise. Besides that, in each resistance the direction of the current was arbitrarily chosen.

\par
In Section~\ref{sec:analysis}, a theoretical analysis of the circuit is presented following two of the most common methods: the mesh method and the nodal method. In Section~\ref{sec:simulation}, the circuit is analysed by simulation. Finally, the  comparison between both types of analysis and the conclusions of this study are in Section~\ref{sec:conclusion}.

