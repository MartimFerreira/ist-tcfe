\section{Theoretical Analysis}
\label{sec:analysis}

\subsection{Mesh Method}
\par
In the mesh method, equations are written for essential meshes, using Kirchhoff's voltage law (KVL), which states that "the directed sum of the potential differences (voltages) around any closed loop is zero". There are also additional equations.

\par
In this method, as well as in the next one, Ohm's Law, $V=RI \Leftrightarrow I = \frac{V}{R}$, was used several times.

% explain how we got to each equation

\begin{equation}
\begin{cases}
R_6 I_3 + R_7 I_3 - V_c + R_4(I_3-I_1) = 0 & \mbox{KVL for mesh $C$}\\
V_a - R_4 (I_3-I_1) - V_b + R_1 I_1 = 0 & \mbox{KVL for mesh $A$}\\
V_c = K_c I_c & \mbox{as given for voltage source $V_c$} \\
I_b = K_b V_b & \mbox{as given for current source $I_b$} \\
I_3 = I_c & \mbox{by inspection in mesh $C$}\\
I_4 = I_d & \mbox{by inspection in mesh $D$}\\
(I_2-I_1) R_3 = V_b & \mbox{Ohm's Law for resistor $R_3$}\\
I_2 = I_b & \mbox{by inspection in mesh $B$}\\
\end{cases}
\end{equation}

\par
\textcolor{red}{\textbf{maybe explaining these equations a little better??}}
\par
\textcolor{red}{\textbf{show system with matrix?}}





\subsection{Nodal Method}
\par
In the nodal method, a set of equation is written based on Kirchhoff's current law (KCL), according to which "the algebraic sum of currents in a network of conductors meeting at a point is zero", which means that the sum of currents flowing into that node is equal to the sum of currents flowing out of that node. Plus, there are some additional equations.

\par
\textcolor{red}{\textbf{mentioning the concept of super node?}}



\begin{equation}
\begin{cases}
V_0 = 0 & \mbox{since $N_0$ is the ground node} \\
V_1-V_0 = V_a & \mbox{by inspection in voltage source $V_a$}  \\
\frac{V_2-V_1}{R_1} + \frac{V_2-V_4}{R_3} + \frac{V_3-V_2}{R_2} = 0 & \mbox{KCL for node $N_2$} \\
\frac{V_3-V_2}{R_2} - (V_2-V_4) K_b = 0 & \mbox{KCL for node $N_3$} \\
\frac{V_0-V_6}{R_6} - \frac{V_6-V_7}{R_7} = 0 & \mbox{KCL for node $N_6$}\\
\frac{V_5-V_4}{R_5} + K_b (V_2 - V_4) - I_d = 0 & \mbox{KCL for node $N_5$}\\
V_4-V_7 = K_c \frac{V_0-V_6}{R_6} & \mbox{by inspection in voltage source $V_c$} \\
\frac{V_2-V_4}{R_3} + \frac{V_5-V_4}{R_5} +  \frac{V_6-V_7}{R_7} - \frac{V_4-V_0}{R_4} - I_d = 0 & \mbox{KCL for node $N_4$} \textcolor{red}{\textbf{     is actually a supernode!}}
\end{cases}
\end{equation}

\par
\textcolor{red}{\textbf{maybe explaining these equations a little better??}}
\par
\textcolor{red}{\textbf{show system with matrix?}}






\subsection{Results}
\par
After solving a linear system of equations with Octave, it was found out that:
\par
\textcolor{red}{\textbf{present values of V1 to V7, Ib, Ic, I1 to I4, Vb and Vc}}


\begin{table}[h]
  \centering
  \begin{tabular}{|c|c|}
    \hline    
    {\bf Name} & {\bf Value [A or V]} \\ \hline
    \input{../mat/octave_node_tab.tex}
  \end{tabular}
  \caption{Operating point. A variable preceded by @ is of type {\em current}
    and expressed in Ampere; other variables are of type {\it voltage} and expressed in
    Volt.}
  \label{tab:op}
\end{table}

