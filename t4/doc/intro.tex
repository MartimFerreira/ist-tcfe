\section{Introduction}
\label {sec:introduction}

% objective
% circuit description
% explain why we decided to call the ground node 0 instead of 4
% circuit image
% --- present the values that were given to us
% --- talk about the significant digits
% sections

% state the learning objective 
With this laboratory assignment we aimed to build an audio amplifier, while maximizing its gain and bandwidth and minimizing both its cost and the value of the lower cutoff frequency.

The input voltage, $V_s$, is sinusoidal and has an amplitude of $10 mV$ and it is in series with resistor $R_s=100 \Omega$, meant to simulate the internal resistance of a voltage source. Then there are the gain and output stages, whose architecture was based on lectures 16 and, mainly, 17. Finally, there is the $8 \Omega$ speaker. In this circuit, the supply voltage source is $V_{cc}=12 V$.

In the gain stage a degenerated common emitter amplifier with a bias circuit and a bypass capacitor were used. In this part there is an NPN transistor.

In the output stage we used a common collector amplifier with a PNP transistor, which will then feed the $8 \Omega$ load.

% modelos usados?
% descrever mais o circuito?
% falar dos componentes e dos valores das grandezas que os caracterizam


\begin{figure}[H] \centering
\includegraphics[width=0.9\linewidth]{rc.pdf}
\caption{The circuit in analysis}
\label{fig:circuit}
\end{figure}


\begin{table}[H]
  \centering
  \begin{tabular}{|c|c|}
    \hline
      \input{resistance_tab.tex}
  \end{tabular}
  \caption{Circuit characteristics}
  \label{tab:resistance}
\end{table}

In sections~\ref{sec:analysis}, ~\ref{sec:simulation} and ~\ref{sec:conclusion}, there are, respectively, the theoretical analysis of the circuit in question (made with the help of \textit{Octave}), the simulation of that same circuit (in which \textit{ngspice} was used) and the conclusion of the report.
