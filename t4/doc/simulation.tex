\section{Simulation Analysis}
\label{sec:simulation}

In this section, $ngspice$ simulation results for the circuit shown above are presented.


\subsection{Gain Response with Frequency}

An AC analysis of the circuit was performed in order to figure out how the gain of the circuit varied with frequency. The resulting graph can be seen in \ref{fig:gain_freq}.

\begin{figure}[H] \centering
\includegraphics[width=0.6\linewidth]{../sim/vo2f.pdf}
\caption{Circuit Gain variation with frequency (in dB)}
\label{fig:gain_freq}
\end{figure}

This type of analysis allows us to determine the maximum gain of the circuit, both cutoff frequencies (frequencies at which the gain becomes 3dB lower than the maximum) and the maximum gain bandwidth. These values, together with the cost of the circuit, can also be used to calculate the merit figure. In order to perform these measurements, the ngspice functions ``MAX'' and ``WHEN'' were used.

\begin{table}[H]
  \centering
  \begin{tabular}{|c|c|}
    \hline
      @gb[i] & -2.32745e-04\\ \hline
@id[current] & 1.004395e-03\\ \hline
@r1[i] & 2.219473e-04\\ \hline
@r2[i] & 2.327451e-04\\ \hline
@r3[i] & -1.07978e-05\\ \hline
@r4[i] & -1.19077e-03\\ \hline
@r5[i] & -1.23714e-03\\ \hline
@r6[i] & 9.688185e-04\\ \hline
@r7[i] & 9.688185e-04\\ \hline
v(1) & 5.179800e+00\\ \hline
v(2) & 4.949130e+00\\ \hline
v(3) & 4.481100e+00\\ \hline
v(4) & 4.981968e+00\\ \hline
v(5) & 8.811951e+00\\ \hline
v(6) & -1.95815e+00\\ \hline
v(7) & -2.94854e+00\\ \hline
v(8) & -1.95815e+00\\ \hline

  \end{tabular}
  \caption{Values obtained}
  \label{tab:resultssim}
\end{table}

A separate analysis was run, with a different script, in order to determine the circuit's output impedance. In this script, the input voltage source was set to have 0 amplitude and the load at the end of the circuit was replace with a 1V amplitude test voltage source. The output impedance obtained was $Z_o = 7.932271 \Omega$.





