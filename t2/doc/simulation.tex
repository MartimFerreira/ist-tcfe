\section{Simulation Analysis}
\label{sec:simulation}

\subsection{Circuit when $t<0$}

Table \ref{tab:comparison1} shows the results of the simulation of the circuit in study for $t<0$ on the left and the theoretical values on the right. For this time condition, $v_s = V_s$.

\begin{table}[H]
  \centering
  \begin{tabular}{|c|c|}
    \hline    
    {\bf Name} & {\bf Value [A or V]} \\ \hline
    @gb[i] & -2.32745e-04\\ \hline
@id[current] & 1.004395e-03\\ \hline
@r1[i] & 2.219473e-04\\ \hline
@r2[i] & 2.327451e-04\\ \hline
@r3[i] & -1.07978e-05\\ \hline
@r4[i] & -1.19077e-03\\ \hline
@r5[i] & -1.23714e-03\\ \hline
@r6[i] & 9.688185e-04\\ \hline
@r7[i] & 9.688185e-04\\ \hline
v(1) & 5.179800e+00\\ \hline
v(2) & 4.949130e+00\\ \hline
v(3) & 4.481100e+00\\ \hline
v(4) & 4.981968e+00\\ \hline
v(5) & 8.811951e+00\\ \hline
v(6) & -1.95815e+00\\ \hline
v(7) & -2.94854e+00\\ \hline
v(8) & -1.95815e+00\\ \hline

  \end{tabular}
  \begin{tabular}{|c|c|}
    \hline    
    {\bf Name} & {\bf Value [A or V]} \\ \hline
    \input{octave_negative_time_tab.tex}
  \end{tabular}
  \caption{Voltages in nodes and currents through branchces for $t<0$ as determined by the simulation (left) and our calculations (right). Values preceded by @ are currents. All currents are in Ampere and all voltages are in Volt}
  \label{tab:comparison1}
\end{table}

Comparing the obtained results from the Theoretical and Simulation Analysis we can see that the all the voltages in nodes and all the currents in branches are consistent for the two different methods. The results are being printed with 12 significant figures and the small difference that can be seen on the last digit of the values are due to approximations that are done when the programs print the results on the tables.

\subsection{Circuit when $t=0$}

Table \ref{tab:comparison2} shows the results of the simulation of the circuit in study for $t=0$ on the left and the theoretical values on the right. For this simulation, the capacitor was replaced by a voltage source with voltage equal to the difference in electric potential between the terminals of the capacitor for $t<0$ (that is, $V_6 - V_8$, with $V_6$ and $V_8$ being taken from the previous analysis). This was done because the electric tension between terminals of a capacitor must always be continuous (since a capacitor takes time to charge and discharge), which means the tension between terminals of the capacitor in $t=0$ must be the same it was in our $t<0$ steady-state analysis (even if $V_6$ and $V_8$ had instantaneous variations). The way to ensure this continuity is by doing the aforementioned replacement.

\begin{table}[H]
  \centering
  \begin{tabular}{|c|c|}
    \hline    
    {\bf Name} & {\bf Value [A or V]} \\ \hline
    \input{op2_tab.tex}
  \end{tabular}
  \begin{tabular}{|c|c|}
    \hline    
    {\bf Name} & {\bf Value [A or V]} \\ \hline
    \input{octave_zero_time_tab.tex}
  \end{tabular}
  \caption{Voltages in nodes and currents through branchces for $t=0$ as determined by the simulation (left) and our calculations (right)}
  \label{tab:comparison2}
\end{table}

In this case we can verify that the results obtained from both methods are one more time consistent - the difference between the voltages values in all nodes for the two methods differ in less that $10^-15$. 
The main difference between the tables are the value of $I_x$ that is simetric. This happens because ngspice , by default, calculates the currents that flows from the positive to negative terminal of the branch and in this case we want to study the opposite direction. 

\subsection{Circuit when $t>0$}

\subsubsection{Natural Solution}

The following graph shows the evolution in time of the natural solution component of the voltage in node 6 according to the simulation that was run. In order to simulate the natural response of the circuit, the voltage source $v_s$ was set to $0$V, and the voltages in nodes 6 and 8 calculated for $t=0$ were used as boundary conditions.

\begin{figure}[H] \centering
\includegraphics[width=0.6\linewidth]{../sim/trans_3.pdf}
\caption{Natural responde in node 6}
\label{fig:natural_sim}
\end{figure}

\subsubsection{Final Solution}

The following graph shows the simulated evolution in time of the voltage in node 6 when the voltage source provides a sinusoidal input with frequency $1$kHz. The evolution in time of the voltage source stimulus can also be seen.

\begin{figure}[H] \centering
\includegraphics[width=0.6\linewidth]{../sim/trans_4.pdf}
\caption{Evolution of stimulus (Vs) and voltage in node 6 (V6) with time}
\label{fig:final_sim}
\end{figure}

\subsubsection{Frequency Response}

The following graphs show the simulated evolution with frequency of the magnitude and phase of the voltage in node 6 as well as of the sinusoidal input from the voltage source for frequencies between $0.1$Hz and $1$MHz.

\begin{figure}[H] \centering
\includegraphics[width=0.6\linewidth]{../sim/mag_4.pdf}
\caption{Magnitude of frequency response in node 6 and stimulus}
\label{fig:mag_sim}
\end{figure}


\begin{figure}[H] \centering
\includegraphics[width=0.6\linewidth]{../sim/phase_4.pdf}
\caption{Phase of frequency response in node 6 and stimulus}
\label{fig:phase_sim}
\end{figure}
