\section{Simulation Analysis}
\label{sec:simulation}

\subsection{Circuit when $t<0$}

Table~\ref{tab:op} shows the simulated operating point results for the circuit
under analysis. Compared to the theoretical analysis results, one notices the
following differences: describe and explain the differences.

\begin{table}[h]
  \centering
  \begin{tabular}{|l|r|}
    \hline    
    {\bf Name} & {\bf Value [A or V]} \\ \hline
    @gb[i] & -2.32745e-04\\ \hline
@id[current] & 1.004395e-03\\ \hline
@r1[i] & 2.219473e-04\\ \hline
@r2[i] & 2.327451e-04\\ \hline
@r3[i] & -1.07978e-05\\ \hline
@r4[i] & -1.19077e-03\\ \hline
@r5[i] & -1.23714e-03\\ \hline
@r6[i] & 9.688185e-04\\ \hline
@r7[i] & 9.688185e-04\\ \hline
v(1) & 5.179800e+00\\ \hline
v(2) & 4.949130e+00\\ \hline
v(3) & 4.481100e+00\\ \hline
v(4) & 4.981968e+00\\ \hline
v(5) & 8.811951e+00\\ \hline
v(6) & -1.95815e+00\\ \hline
v(7) & -2.94854e+00\\ \hline
v(8) & -1.95815e+00\\ \hline

  \end{tabular}
  \caption{Operating point. A variable preceded by @ is of type {\em current}
    and expressed in Ampere; other variables are of type {\it voltage} and expressed in
    Volt.}
  \label{tab:t<0_sim}
\end{table}


\subsection{Circuit when $t=0$}

Table~\ref{tab:op} shows the simulated operating point results for the circuit
under analysis. Compared to the theoretical analysis results, one notices the
following differences: describe and explain the differences.

\begin{table}[h]
  \centering
  \begin{tabular}{|l|r|}
    \hline    
    {\bf Name} & {\bf Value [A or V]} \\ \hline
    \input{../sim/op2_tab}
  \end{tabular}
  \caption{Operating point. A variable preceded by @ is of type {\em current}
    and expressed in Ampere; other variables are of type {\it voltage} and expressed in
    Volt.}
  \label{tab:t=0_sim}
\end{table}


\subsection{Circuit when $t>0$}

\subsubsection{Natural Solution}

Figure~\ref{fig:trans} shows the simulated transient analysis results for the
circuit under analysis. Compared to the theoretical analysis results, one
notices the following differences: describe and explain the differences.

\begin{figure}[h] \centering
\includegraphics[width=0.6\linewidth]{../sim/trans_3.pdf}
\caption{Natural responde in node 6}
\label{fig:natural_sim}
\end{figure}

\subsubsection{Final Solution}

Figure~\ref{fig:acm} shows the magnitude of the frequency response for the
circuit under analysis. Compared to the theoretical analysis results, one
notices the following differences: describe and explain the differences.

\begin{figure}[h] \centering
\includegraphics[width=0.6\linewidth]{../sim/trans_4.pdf}
\caption{Evolution of stimulus (Vs) and voltage in node 6 (V6) with time}
\label{fig:final_sim}
\end{figure}

\lipsum[1-1]

\subsubsection{Frequency Response}

Figure~\ref{fig:acp} shows the magnitude of the frequency response for the
circuit under analysis. Compared to the theoretical analysis results, one
notices the following differences: describe and explain the differences.

\begin{figure}[h] \centering
\includegraphics[width=0.6\linewidth]{../sim/mag_4.pdf}
\caption{Magnitude of frequency response in node 6 and stimulus}
\label{fig:mag_sim}
\end{figure}


\begin{figure}[h] \centering
\includegraphics[width=0.6\linewidth]{../sim/phase_4.pdf}
\caption{Phase of frequency response in node 6 and stimulus}
\label{fig:phase_sim}
\end{figure}
