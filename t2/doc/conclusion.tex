
\section{Conclusion}
\label{sec:conclusion}

\begin{table}[H]
  \centering
  \begin{tabular}{|c|c|}
    \hline    
    {\bf Name} & {\bf Value [A or V]} \\ \hline
    \input{octave_negative_time_tab.tex}
  \end{tabular}
  \begin{tabular}{|c|c|}
    \hline    
    {\bf Name} & {\bf Value [A or V]} \\ \hline
    @gb[i] & -2.32745e-04\\ \hline
@id[current] & 1.004395e-03\\ \hline
@r1[i] & 2.219473e-04\\ \hline
@r2[i] & 2.327451e-04\\ \hline
@r3[i] & -1.07978e-05\\ \hline
@r4[i] & -1.19077e-03\\ \hline
@r5[i] & -1.23714e-03\\ \hline
@r6[i] & 9.688185e-04\\ \hline
@r7[i] & 9.688185e-04\\ \hline
v(1) & 5.179800e+00\\ \hline
v(2) & 4.949130e+00\\ \hline
v(3) & 4.481100e+00\\ \hline
v(4) & 4.981968e+00\\ \hline
v(5) & 8.811951e+00\\ \hline
v(6) & -1.95815e+00\\ \hline
v(7) & -2.94854e+00\\ \hline
v(8) & -1.95815e+00\\ \hline

  \end{tabular}
  \caption{Extra results obtained from solving the equation system due to us explicitly showing equations (5) to (8). These will be useful during the method comparison.}
  \label{tab:mesh}
\end{table}

\begin{table}[H]
  \centering
  \begin{tabular}{|c|c|}
    \hline    
    {\bf Name} & {\bf Value [A or V]} \\ \hline
    \input{octave_zero_time_tab.tex}
  \end{tabular}
  \begin{tabular}{|c|c|}
    \hline    
    {\bf Name} & {\bf Value [A or V]} \\ \hline
    \input{op2_tab.tex}
  \end{tabular}
  \caption{Extra results obtained from solving the equation system due to us explicitly showing equations (5) to (8). These will be useful during the method comparison.}
  \label{tab:mesh}
\end{table}


On the whole, the objectives set for this laboratory assignment were successfully met.
In both the simulation and the analysis part, the voltages for all nodes and the currents for all branches were calculated for $t<0$. $v_{6_n}(t)$ and $v_{6_n}(t)$,the natural and the forced solutions, were also determined and, finally, the total solution for $V_6(t)$ when $t>0$, as well as the time and frequency responses for the voltage in certain nodes. Although the results obtained through these methodologies were not exactly the same, they were overall quite similar.
% desenvolver mais?



