\section{Conclusion}
\label{sec:conclusion}

\begin{table}[hbt!]
  \centering
  \begin{tabular}{|c|c|}
    \hline    
    {\bf Name} & {\bf Value [A or V]} \\ \hline
    \input{octave_negative_time_tab.tex}
  \end{tabular}
  \begin{tabular}{|c|c|}
    \hline    
    {\bf Name} & {\bf Value [A or V]} \\ \hline
    @gb[i] & -2.32745e-04\\ \hline
@id[current] & 1.004395e-03\\ \hline
@r1[i] & 2.219473e-04\\ \hline
@r2[i] & 2.327451e-04\\ \hline
@r3[i] & -1.07978e-05\\ \hline
@r4[i] & -1.19077e-03\\ \hline
@r5[i] & -1.23714e-03\\ \hline
@r6[i] & 9.688185e-04\\ \hline
@r7[i] & 9.688185e-04\\ \hline
v(1) & 5.179800e+00\\ \hline
v(2) & 4.949130e+00\\ \hline
v(3) & 4.481100e+00\\ \hline
v(4) & 4.981968e+00\\ \hline
v(5) & 8.811951e+00\\ \hline
v(6) & -1.95815e+00\\ \hline
v(7) & -2.94854e+00\\ \hline
v(8) & -1.95815e+00\\ \hline

  \end{tabular}
  \caption{Extra results obtained from solving the equation system due to us explicitly showing equations (5) to (8). These will be useful during the method comparison.}
  \label{tab:mesh}
\end{table}

\begin{table}[hbt!]
  \centering
  \begin{tabular}{|c|c|}
    \hline    
    {\bf Name} & {\bf Value [A or V]} \\ \hline
    \input{octave_zero_time_tab.tex}
  \end{tabular}
  \begin{tabular}{|c|c|}
    \hline    
    {\bf Name} & {\bf Value [A or V]} \\ \hline
    \input{op2_tab.tex}
  \end{tabular}
  \caption{Extra results obtained from solving the equation system due to us explicitly showing equations (5) to (8). These will be useful during the method comparison.}
  \label{tab:mesh}
\end{table}

In this laboratory assignment the objective of analysing an RC circuit has been
achieved. Static, time and frequency analyses have been performed both
theoretically using the Octave maths tool and by circuit simulation using the
Ngspice tool. The simulation results matched the theoretical results
precisely. The reason for this perfect match is the fact that this is a
straightforward circuit containing only linear components, so the theoretical
and simulation models cannot differ. For more complex components, the
theoretical and simulation models could differ but this is not the case in this
work.
