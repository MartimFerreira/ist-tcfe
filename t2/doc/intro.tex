\section{Introduction}
\label{sec:introduction}

% objective
% circuit description
% explain why we decided to call the ground node 0 instead of 4
% circuit image
% --- present the values that were given to us
% --- talk about the significant digits
% --- sections

% state the learning objective 
The aim of this laboratory assignment is to study the RC circuit in Figure~\ref{fig:rc}. It contains two voltage sources ($V_d$ a linear current controlled one and $V_S$, an independent one, which supplies voltage $V_S$ for $t\leq0$ and $\sin (2 \pi f t)$ for $t>0$), one current source ($I_b$ a linear voltage controlled one), resistors ($R_1$, $R_2$, $R_3$, $R_4$, $R_5$, $R_6$ and $R_7$) and a capacitor ($C$).
With regard to the identification of the nodes it was decided that the ground node would hereafter be referred to as node 0. Node 4 was also an option but because in \textit{Ngspice} 0 is the ground node, this notation was also used in the theoretical analysis for reasons of convenience and coherence.


\lipsum[1-1]


\begin{figure}[h] \centering
\includegraphics[width=0.6\linewidth]{rc.pdf}
\caption{The circuit in analysis}
\label{fig:rc}
\end{figure}

% *** tabela com os valores gerados

% falar outra vez dos algarismos significativos?

% secções

In Section~\ref{sec:analysis}, a theoretical analysis of the circuit is
presented. In Section~\ref{sec:simulation}, the circuit is analysed by
simulation, and the results are compared to the theoretical results obtained in
Section~\ref{sec:analysis}. The conclusions of this study are outlined in
Section~\ref{sec:conclusion}.
